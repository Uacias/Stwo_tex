\documentclass{article}
\usepackage[utf8]{inputenc}
\usepackage[T1]{fontenc}
\usepackage{cmap}
\usepackage{amsmath}
\usepackage{graphicx}
\usepackage{geometry}
\geometry{a4paper}

\title{Propozycja Tematu Pracy Magisterskiej: Wykorzystanie zk-Proof (STEW) do Nowej Metody Autentykacji}
\author{Filip Gumuła}
\date{\today}

\begin{document}

\maketitle

Celem pracy magisterskiej jest opracowanie nowej metody autentykacji użytkowników w systemach informatycznych z wykorzystaniem zero-knowledge proofs (zk-Proof), w szczególności algorytmu STARK. Tradycyjne metody autentykacji, oparte na przesyłaniu hasła na serwer, są podatne na różnorodne ataki. Praca ta proponuje innowacyjne podejście, w którym użytkownik udowadnia znajomość swojego hasła bez konieczności przesyłania go wprost.

W ramach tej pracy, szczególną uwagę poświęcono Stwo Proverowi, który jest szybkim, otwartoźródłowym proverem, 
który implementuje przełomowy protokół Circle STARK. Dzięki temu, Stwo odblokowuje pełny potencjał wysoce 
efektywnego liczby pierwszej Mersenne'a M31, przynosząc korzyści w przestrzeni zk-proof i blockchain. 
Kluczowe cechy Stwo Prover, takie jak Circle STARK, otwartoźródłowość, skalowalność i kompatybilność, 
są istotne dla opracowywanej metody autentykacji, oferując niezrównaną wydajność dowodzenia. 

Podsumowując, Stwo Prover i jego wykorzystanie w kontekście zero-knowledge proofs stanowi kluczowy element tej pracy magisterskiej, oferując nową, bezpieczniejszą i bardziej efektywną metodę autentykacji użytkowników w systemach informatycznych.

\section{Wstępny Plan Pracy}
\subsection{Rozdział 1: Wprowadzenie}
\begin{itemize}
    \item Cel pracy
    \item Zakres pracy
    \item Struktura pracy
\end{itemize}

\subsection{Rozdział 2: Teoretyczne Podstawy zk-Proof}
\begin{itemize}
    \item Wprowadzenie do kryptografii
    \item Zero-Knowledge Proofs: definicje i właściwości
    \item Algorytm STARK: zasady działania
    \item Stwo Prover: opis i właściwości
    \item Przegląd istniejących rozwiązań i ich ograniczenia
\end{itemize}

\subsection{Rozdział 3: Projekt Systemu Autentykacji}
\begin{itemize}
    \item Wymagania systemowe
    \item Architektura systemu
    \item Wybór algorytmów kryptograficznych: Pedersen/Poseidon
    \item Mechanizm challenge-response
\end{itemize}

\subsection{Rozdział 4: Implementacja}
\begin{itemize}
    \item Środowisko programistyczne i narzędzia
    \item Implementacja po stronie przeglądarki: obliczanie hash i generowanie zk-Proof
    \item Implementacja po stronie serwera: weryfikacja zk-Proof
    \item Integracja z istniejącymi systemami autentykacji
\end{itemize}

\subsection{Rozdział 5: Testowanie i Walidacja}
\begin{itemize}
    \item Scenariusze testowe
    \item Bezpieczeństwo systemu
    \item Wydajność i skalowalność
    \item Porównanie z tradycyjnymi metodami autentykacji
\end{itemize}

\subsection{Rozdział 6: Dyskusja}
\begin{itemize}
    \item Zalety i wady proponowanego rozwiązania
    \item Możliwości dalszego rozwoju
    \item Potencjalne zagrożenia i sposoby ich mitigacji
\end{itemize}

\subsection{Rozdział 7: Podsumowanie}
\begin{itemize}
    \item Wnioski
    \item Osiągnięcia pracy
    \item Przyszłe kierunki badań
\end{itemize}

\section*{Bibliografia}

\section*{Załączniki}
Opis Wybranych Narzędzi i Bibliotek:
\begin{itemize}
    \item STEW
    \item Cartridge Oasis
    \item Stone Prover
    \item State Channel Framework
    \item Rust Programming Language
    \item WebAssembly
    \item Yew Framework
\end{itemize}

\end{document}